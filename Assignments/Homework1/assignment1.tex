\documentclass[10pt,onecolumn,journal,draftclsnofoot]{IEEEtran}
\usepackage[margin=0.75in]{geometry}
\renewcommand{\familydefault}{\rmdefault}

\usepackage{listings}
\usepackage{color}

\definecolor{dkgreen}{rgb}{0,0.6,0}
\definecolor{gray}{rgb}{0.5,0.5,0.5}
\definecolor{mauve}{rgb}{0.58,0,0.82}

\lstset{frame=tb,
  language=Bash,
  aboveskip=3mm,
  belowskip=3mm,
  showstringspaces=false,
  columns=flexible,
  basicstyle={\small\ttfamily},
  numbers=none,
  numberstyle=\tiny\color{gray},
  keywordstyle=\color{blue},
  commentstyle=\color{dkgreen},
  stringstyle=\color{mauve},
  breaklines=true,
  breakatwhitespace=true,
  tabsize=3
}

\begin{document}

\begin{titlepage}
\title{CS 444 - Operating Systems II- Assignment 1}
\author
{\IEEEauthorblockN{Christian Mello, Daniel Stewart, Shengpei Yuan\\
Group 12-06\\}
\IEEEauthorblockA{
Oregon State University\\
Computer Science 444\\
Spring 2017
}}
	\maketitle
	\vspace{4cm}
	\begin{abstract}
		\noindent This document outlines assignment 1 for Operating Systems II.
		There are two parts to the assignment, setting up a Qemu based Yocto environment, and a concurrency exercise using producers and consumers.
	\end{abstract}

\end{titlepage}

\newpage

\section{Concurrency}
\par
The main point of this assignment was to familiarize ourselves again with the concept of both concurrency and thinking in parallel.
The problem calls for the creation of multiple threads, which all carry out one of two tasks, either producing or consuming. 
This means we have to think about what a consuming thread is doing when there is nothing to consume, and what threads are doing when another thread is either adding something or taking something away from the shared buffer.
\\
\par
We approached the problem by first dealing with the generation of random numbers. 
As the problem states, it needs to use both rdrand x86 ASM when supported, and Mersenne Twister when not. 
This is done with an if statement in the producer and consumer functions themselves. 
Each one calls another function, which returns a 1 if it can use ASM, and a 0 if not.
I suppose this could have also been done off a global variable, but this way works.
After this we implement the buffer to hold 32 objects at time, which are random integers, along with a lock and a condition for empty and full.
We then only made 2 threads to start, one each for consume and produce. 
The contents of the buffer get printed out after each thread action, along with the time that they sleep for. 
The threads will also lock any other threads out of using the buffer while they are.
A thread trying to use the locked buffer will just wait until it becomes available again.
After making sure that 2 threads worked, implementing more wasn't hard, as it just required making an array to hold each thread.
We also added a command line input variable for the number of threads desired, which can't be less than 2. 
It splits the desired number into consumer and producer threads using the mod operator. 
\\
\par
There wasn't a lot that we thought needed to be done to make sure that our implementation was running the way it was intended to.
It prints out the wait time for each thread, and the numbers in the buffer after each thread access it. 
So, running the program with different amounts of threads we can see what is going on.
Specifically if the program is run with an odd number of threads, like 3. 
Since this creates 2 producer and 1 consumer thread, it tends to fill the buffer up fairly quick compared to only 2 threads, since 2 threads are adding new integers to it.
We also ran it on our own machines, which should use rdrand, and on os-access, which won't be able to use it, and it works for both.
\\
\par
The things learned during this assignment were mostly ones forgotten from Operating Systems I, or that hadn't been used in a while, such as setting up pthreads and concurrency problems. 

\newpage

\section{QEMU Flag Explanation}

\begin{itemize}

\item -gdb tcp::5726
\begin{itemize}
\item{Tells qemu to wait for a communication on our specified gdb device, which we have hosted on socket 5726. It won't run until connected.}
\end{itemize}
\item -S
\begin{itemize}
\item{Tells qemu not to start the cpu on system startup.}
\end{itemize}
\item -nographic
\begin{itemize}
\item{Completely disables graphical output for qemu.}
\end{itemize}
\item -kernel linux-yocto-3.14/arch/x86/boot/bzImage
\begin{itemize}
\item{Specifies a kernel image for the session. This is the one we compiled for the assignment.}
\end{itemize}
\item -drive file=core-image-lsb-sdk-qemux86.ext3,if=virtio
\begin{itemize}
\item{Tells qemu a drive to boot from. The file= argument lets us use a disk image file, and creates one if it doesn't exist.}
\end{itemize}
\item -enable-kvm
\begin{itemize}
\item{Enables KVM full virtualization support.}
\end{itemize}
\item -net none
\begin{itemize}
\item{Tells qemu that no network devices should be configured.}
\end{itemize}
\item -usb
\begin{itemize}
\item{Enables the usb driver.}
\end{itemize}
\item -localtime
\begin{itemize}
\item{Uses localtime instead of UTC.}
\end{itemize}
\item --no-reboot
\begin{itemize}
\item{Exits the system instead of rebooting when a reboot command is run.}
\end{itemize}
\item --append "root=/dev/vda rw console=ttyS0 debug"
\begin{itemize}
\item{Sets the current command line options for the kernel boot.}
\end{itemize}

\end{itemize}

\newpage

\section{Command Log}

\begin{lstlisting}
// command_log.sh
# Logging into os-class server:

ssh flip.engr.oregonstate.edu
ssh os-class.engr.oregonstate.edu

# Creating team directory:

cd /
cd scratch
cd spring2017
cd 10-01
ls
cd ..
mkdir 12-06
cd 12-06

# Cloning yocto repository and checking out 3.14 tag:

git clone git://git.yoctoproject.org/linux-yocto-3.14
cd linux-yocto-3.14
git checkout -b v3.14.26

# Sourcing proper profile:

source /scratch/opt/environment-setup-i586-poky-linux

# Copying default config:

cp /scratch/spring2017/files/config-3.14.26-yocto-qemu .config

# menuconfig, changing the version string to our team:

make menuconfig

# Building the kernel:

make -j4 all

# Running gdb in terminal 2:

source /scratch/opt/environment-setup-i586-poky-linux
cd /scratch/spring2017/12-06
gdb

# Copying sample bzImage and filesystem:

cp /scratch/spring2017/files/bzImage-qemux86.bin .
cp /scratch/spring2017/files/core-image-lsb-sdk-qemux86.ext3 .

# Running sample kernel build:

qemu-system-i386 -gdb tcp::5726 -S -nographic -kernel bzImage-qemux86.bin -drive file=core-image-lsb-sdk-qemux86.ext3,if=virtio -enable-kvm -net none -usb -localtime --no-reboot --append "root=/dev/vda rw console=ttyS0 debug"

# Running our kernel build:

qemu-system-i386 -gdb tcp::5726 -S -nographic -kernel linux-yocto-3.14/arch/x86/boot/bzImage -drive file=core-image-lsb-sdk-qemux86.ext3,if=virtio -enable-kvm -net none -usb -localtime --no-reboot --append "root=/dev/vda rw console=ttyS0 debug"

# Logging out:

exit
\end{lstlisting}

\section{Work Log}
\par
4/12/2017 - Discussed assignment 1. Set up the group directory on os-class and ran the commands to set up the Qemu environment.
\par
4/19/2017 - Wrote the bulk of the concurrency problem code. Everything finished except for inputting the number of threads.
\par
4/21/2017 - Finished up the concurrency code by making it multithreaded and adding the command line input. Wrote the LaTeX document, and then created a makefile for the concurrency code. Combined the makefile for the LateX document with that of the concurrency code and debugged issues with building it correctly. 



\end{document}
